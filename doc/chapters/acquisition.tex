%%% -*- TeX-master: "../main" -*-
\chapter{Dataset Acquisition}
\label{chapter:acquisition}

\section{Recording Setup}
\label{sec:setup}

\section{Camera Calibration}
\label{sec:calibration}

\section{Labeling}
\label{sec:labeling}

The data is first manually labeled with a specifically written labeling program that allows marking the beginning and end of a gesture with a precision of about 0.1 seconds.

In the training phase, each event is labeled with a gesture or an extra no-gesture label.
To ensure that each timeslice lies wholly within a gesture's boundaries, we keep a buffer of 100 milliseconds to the beginning as well as the end of a gesture.
Events in this buffer zone are marked with the no-gesture label.
This prevents two problems.
First, the precision of the manual start and end markers is in the 100ms range, so this buffer ensures that the events truly belong to the gesture.
Second
Without this correction we would label a timeslice with a gesture even though only the first few events actually belong to this gesture and
